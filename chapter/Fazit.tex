\section{Fazit}
\label{sec:Fazit}

In Anbetracht der aufgezeigten Unterschiede ist fest zuhalten, dass jede
Programmiersprache über ihre eigenen Stärken verfügt. Ein Programmierer, 
welche die Herausforderung in der Speicherverwaltung und Fehlerbehandlung von C
bewältigen kann, profitiert von der Performance und Hardware nahen 
Programmierung.
Rust bewältigt die Herausforderung von C für den Programmierer durch dem strikten
Typisierung und dem Ownership-Modell. Dadurch ist von einem Rust Programmierer 
jedoch verlangt neue Konzepte zu erlernen. 
Go wiederum setzt auf eine einfache Syntax und bietet Möglichkeiten parallele
Programme zu schreiben. 
Sowohl Rust als auch Go bieten integrierte Build-Tooling an welche Aufgaben, wie
Paketmanaging und Building vereinfachen.

Insgesamt lässt sich festhalten, dass die Wahl der Programmiersprache stark von 
den Anforderungen des Projekts und den Präferenzen des Entwicklerteams abhängt. 
C bleibt für performante, hardwarenahe Anwendungen relevant, Go eignet sich 
besonders für einfache, parallele und cloudbasierte Systeme, während Rust eine 
moderne Alternative für sichere und effiziente Softwareentwicklung darstellt. 
Die Weiterentwicklung von Konzepten aus C in Go und Rust zeigt, wie wichtig 
Innovation und Standardisierung für die Zukunft der Programmierung sind.