\section{Parallelität}
\label{sec:Parallelität}

Mit der Zunahme an Kernen in modernen Prozessoren macht die parallele Gestaltung 
von Quellencode immer relevanter. Jedoch birgt die parallele Programmierung
auch einige Herausforderungen für den Programmierer mit sich. Programmiersprachen
können bei diesen Herausforderungen unterstützen oder dem Programmierer die Arbeit
erschweren. In diesem Abschnitt wird herausgearbeitet, wie Rust und Go den
Programmierer besser bei seiner Arbeit unterstützen als C.

\subsection{Parallelität in C}
\label{subsec:Parallelität in C}

Es wird begonnen die grundlegende Gestalung von Parallelität in C zu erläutern.
Zu diesem Zweck wird ein minimales Beispielprogramm in C erstellt, welches im 
Anhang \ref{anhang:vollstaendiger_quelltext} zu finden ist. 

In \cref{fig:datentyp_in_c} wird gezeigt, wie in C Threads initialisiert werden
und ihre Referenzen in einem Array gespeichert werden. In C gibt es nicht die 
Möglichkeit einen Thread ohne eine Referenz zu verwenden. Des Weiteren ist zu 
bemerken, dass in C Daten nur typenlos übergeben werden können und auch Resultate
können nur typenlos erhalten werden. Dies bedeutet, dass der Programmierer
selbst dafür sorgen muss, dass die Daten in der richtigen Form vorliegen. 
\autocite{hawthorneLanguageComparisonParallel}

\begin{figure}[htp]
    \centering
    \lstinputlisting[
        language=C,
        linerange={18-28},
        caption=Beispiel für die Verwendung Kollektionen zur Speicherung von Threads in C,
        label=fig:datentyp_in_c
    ]{img/parallelitaet_in_c.c}
\end{figure}

In \cref{fig:mutex_in_c} zeigt, dass um Synchronisation zu ermöglichen, muss
ein Mutex verwendet werden, um einen Codeabschnitt zu schützen. Prinzipell sind 
auch andere Synchronisationsmechanismen möglich, jedoch müssen diese vom 
Programmierer manuell angewendet werden. 

\begin{figure}[htp]
    \centering
    \lstinputlisting[
        language=C,
        linerange={10-14},
        caption=Beispiel für die Verwendung von Mutex in C,
        label=fig:mutex_in_c
    ]{img/parallelitaet_in_c.c}
\end{figure}


\subsection{Erweiterung in Go}
\label{subsec:Erweiterung in Go}
Im Kontrast zu C bietet Go mehr Unterstützung für das Programmieren mit Threads.
So zeigt \cref{fig:goroutinen}, dass in Go das Schlagewort \texttt{go} verwendet 
wird, um eine Goroutine zu starten, welche Equivalent zu einem Thread ist.

\begin{figure}[htp]
    \centering
    \lstinputlisting[
        language=go,
        linerange={16-23},
        caption=Beispiel für die Verwendung von Goroutinen in Go,
        label=fig:goroutinen
    ]{img/parallelitaet_in_go.go}
\end{figure}

Zur Synchronisation und Kommunikation zwischen Goroutinen bietet Go
Kanäle an. In \cref{fig:go_kanaele} kann gesehen werden, wie Daten über einen 
Kanal gesendet werden können. Dadurch das Kanäle in Go einen Datentyp erhalten,
hat der Programmierer mehr Sicherheit und eine leichere Handhabung. Des Weiteren
kann man mit Kanälen flexibeler die Kommunikation zwischen Goroutinen gestalten.
Dadruch dass Kanäle eine Buffergröße haben können, kann sowohl ein Mutex 
nachgestellt werden, wenn eine Buffergröße von 1 verwendet wird, als auch eine 
beabsichtigte Toleranz eingebaut werden, wie weit eine oder mehrere Goroutinen 
vorarbeiten dürfen, bevor sie auf das Lesen auf der anderen Seite des Kanals 
warten. \autocite{hawthorneLanguageComparisonParallel}

\begin{figure}[htp]
    \centering
    \lstinputlisting[
        language=go,
        linerange={25-28, 8-14},
        caption=Beispiel für die Verwendung von Kanälen in Go,
        label=fig:go_kanaele,
    ]{img/parallelitaet_in_go.go}
\end{figure}

\subsection{Erweiterung in Rust}
\label{subsec:Erweiterung in Rust}

Rust ändert die Grundsätzliche Verwendung von Threads nicht, deswegen ist eine 
ähnlichkeit der Struktur in \cref{fig:datentyp_in_c} und \cref{fig:rust_threads}
zu erkennen. Jedoch bietet Rust eine bessere Typensicherheit, da die Referenzen
von Daten, welche den Threads übergeben werden, nicht nur typisiert sind, sondern
auch der Arc-Typ verwendet. Dieser Arc-Typ ist ein Thread-sicherer 
Referenzzähler, welcher es ermöglicht, dass mehrere Threads auf die gleichen Daten
zugreifen können, ohne dass der Programmierer sich um die Synchronisation kümmern
muss. Dies wird bewekstäligt, inder der Arc-Typ erwartet, dass im ein Datentyp 
übergeben wird, welcher den \texttt{Send}-Trait und \texttt{Sync} implementiert. 
Diese Traits garantieren, dass der Datentyp sicher zwischen Threads 
übertragen werden kann und dass er von mehreren Threads gleichzeitig gelesen aber
nicht beschrieben werden kann.
\autocite{pfosiComparisonConcurrencyRust,IntroductionRustExample}

\begin{figure}[htp]
    \centering
    \lstinputlisting[
        language=Rust,
        linerange={9-26},
        caption=Beispiel für die Verwendung von Threads in Rust,
        label=fig:rust_threads,
    ]{img/parallelitaet_in_rust.rs}
\end{figure}

Des Weiteren bietet Rust die Möglichkeit, Daten über Kanäle zu senden, ähnlich
wie in Go mit der besonderheit, dass Rust Kanäle unidirektional sind im Gegensatz
zu Gos bidirektionale Kanäle. \autocite{ChannelsRustExample}

Ändere den Text