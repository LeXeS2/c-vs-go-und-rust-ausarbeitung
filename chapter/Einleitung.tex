\section{Einleitung}
\label{sec:Einleitung}

Die Programmiersprache C wurde im Jahre 1972 in der dritten Generation 
der Programmiersprachen entwickelt, in dieser lag der Fokus darauf, dass 
Programmiersprachen von der unterliegenden Hardware abstrahierten um so Hardware 
unabhängig programmieren zu können. \autocite{sharmaShortCommunicationComputer2020}

Seid dem haben sich die Anforderungen an eine Programmiersprache geändert.
So entstand Rust mit dem Ziel schnelle Programme zu erzeugen und den Programmierer
dabei zu unterstützen Spiechersicher zu sein. \autocite{IntroductionRustExample}
Zum anderen entwickelte Google die Sprache Go um Parallele Programme einfach und
verständlich programmieren zu können. \autocite{GoProgrammingLanguage}

In dieser Arbeit soll darauf eingegangen werden, wie Rust und Go sich die 
Konzepte von C weiterentwickeln, dabei wird auf vier Eigenschaften eingangen:
\begin{itemize}
    \item Speicherverwaltung,
    \item Parallelität,
    \item Build Tooling und
    \item Fehlerbehandlung
\end{itemize}
