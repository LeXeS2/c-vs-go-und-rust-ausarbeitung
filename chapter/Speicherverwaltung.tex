\section{Speicherverwaltung}
\label{sec:Speicherverwaltung}
In modernen Computersystemen ist Speicher ein vom Prozessor direkt zugreifbarer Hochgeschwindigkeitsspeicher, 
meist in Form von DRAM, der Programme und deren Daten speichert\autocite{OperatingSystemConcepts}. 
Da physischer Speicher schnell, aber begrenzt und flüchtig ist, wurde mit wachsender Programmgrösse 
und Mehrtasking-Betriebssystemen virtueller Speicher eingeführt.

Virtueller Speicher trennt den vom Programm wahrgenommenen logischen Adressraum vom physischen Speicher 
und vermittelt jedem Prozess den Eindruck eines grossen, zusammenhängenden Adressraums\autocite{OperatingSystemConcepts}. 
Durch Techniken wie Paging, Segmentierung und Demand Paging können Speicherbedarf und Sicherheit verbessert 
sowie Prozesse isoliert werden\autocite{IntroductionandOverviewoftheMulticsSystem}.

Die Speicherverwaltung koordiniert physische und virtuelle Speicherressourcen, weist Speicher zu, schützt Prozesse, 
wandelt Adressen um und lagert selten genutzte Daten aus, um eine effiziente und sichere Programmausführung in 
Mehrprozessumgebungen zu gewährleisten\autocite{Operating}.
\subsection{Speicherverwaltung basierend auf Programmiersprachen}
\label{sec:Speicherverwaltung basierend auf Programmiersprachen}
Speicherverwaltung ist nicht nur ein interner Mechanismus des Betriebssystems, sondern auch mit 
Programmiersprachen verknüpft. Aus der Perspektive der Implementierung von Programmiersprachen lassen sich 
zwei grundlegende Ansätze unterscheiden: manuelle Speicherverwaltung und automatische Speicherverwaltung.

\subsubsection{Manuelle Speicherverwaltung}
\label{sec:Manuelle Speicherverwaltung}

Manual Memory Management bezeichnet ein Verwaltungsmodell, bei dem die Zuweisung und Freigabe von Speicher 
während der Programmausführung explizit durch den Programmierer gesteuert werden und nicht automatisch durch 
die Laufzeitumgebung oder das Betriebssystem erfolgen\autocite{TheCProgrammingLaunguage}. In diesem Modell bestimmt der 
Entwickler mithilfe spezieller Funktionen zur Speicherzuweisung und -freigabe direkt den Lebenszyklus von 
Speicherblöcke im Heap. Das bedeutet, dass die Beschaffung, Nutzung und Freigabe von Speicher vollständig 
von der Logik und den Entscheidungen des Programmierers abhängen; Compiler und Laufzeitumgebung führen 
keine automatische Bereinigung nicht mehr genutzter Objekte durch.

In C verwendet man wie malloc(), calloc(), realloc() und free(), und 
in C++ wie new und delete verwendet werden, um den Speicherlebenszyklus 
im Heap direkt zu steuern. Dieses Verfahren bietet ein hohes Mass an 
Flexibilität und Vorhersehbarkeit: Entwickler können exakt festlegen, 
wann Speicher zugewiesen und freigegeben wird, und durch benutzerdefinierte 
Allokatoren oder Speicherpools die Leistung weiter optimieren\autocite{SmashingtheStackforFunandProfit}.

Allerdings bringt manuelle Verwaltung erhebliche Komplexität und Risiken mit 
sich. Da eine automatisierte Speicherbereinigung fehlt, führt das Versäumnis, 
nicht mehr benötigten Speicher rechtzeitig freizugeben, zu Speicherlecks, die 
bei langfristiger Ausführung die Systemressourcen erschöpfen können. Das Zugreifen 
auf bereits freigegebenen Speicher verursacht hängende Zeiger, was zu 
Programmabstürzen oder gar Sicherheitslücken führen kann\autocite{Seacord}. 
Darüber hinaus kann ein fehlerhaftes doppeltes Freigeben von Speicher die 
Heap beschädigen, während Pufferüberläufe zusätzliche Sicherheitsrisiken darstellen.

Trotz dieser Gefahren bleibt die manuelle Speicherverwaltung in Szenarien, 
die höchste Anforderungen an Leistung und Ressourcenkontrolle stellen, unverzichtbar 
wie in Betriebssystemkernen, eingebetteten Systemen oder im Hochleistungsrechnen.
\subsubsection{Automatische Speicherverwaltung}
\label{sec:Automatische Speicherverwaltung}

Automatische Speicherverwaltung bedeutet, dass nicht mehr genutzte Speicherobjekte 
durch einen Garbage Collection Mechanismus automatisch freigegeben werden, sodass 
der Programmierer den Speicher nicht manuell freigeben muss.

Speicherverwaltungsstrategien spiegeln tiefgreifend die Unterschiede im Designhintergrund, der 
Entwicklungszeit und den Zielanwendungsszenarien wider.

\subsection{Typische Speicherverwaltungsfehler in C}
\label{sec:Typische Speicherverwaltungsfehler in C}

Die Programmiersprache C wurde 1972 von Dennis Ritchie in den Bell 
Laboratories entwickelt, um portablen und effizienten Systemcode für 
das Unix Betriebssystem zu schreiben. Daher bietet C als frühe systemnahe 
Programmiersprache direkte Zugriff auf den Speicher und ein hohes Mass an Flexibilität. 
Gleichzeitig bringt der vollständig manuelle Speicherverwaltungsmechanismus von C 
jedoch zahlreiche potenzielle Risiken mit sich\autocite{TheCProgrammingLaunguage}.

Im vollständig manuellen Speicherverwaltungsmodell liegen Anforderung, Nutzung und 
Freigabe von Speicher vollständig in der Verantwortung des Entwicklers, während die 
Sprache selbst nahezu keine Laufzeitprüfungen oder automatischen Schutzmechanismen 
bereitstellt. Dies führt dazu, dass C Programme besonders anfällig für 
Speicherverwaltungsfehler sind,insbesondere in komplexen Projekten oder mit 
mehreren Entwicklern arbeiten zusammen.

Im Folgenden werden einige typische Fehlerarten aufgeführt, die bis heute eine 
wesentliche Ursache für Probleme in Systemsoftware und Sicherheitslücken darstellen.

\subsubsection{Dangling Pointers}
\label{sec:Dangling Pointers}
Ein hängender Zeiger ist ein Zeiger, der auf bereits freigegebenen Speicher verweist. 
Da C keine Überprüfung der Gültigkeit von Zeigern durchführt, führt der weitere Zugriff 
auf diesen Speicherblock nach der Freigabe zu undefiniertem Verhalten, was sich in 
Datenkorruption oder Programmabstürzen äussern kann betonen bei der Implementierung eines Speichermanagers, dass die Weiterverwendung eines 
freigegebenen Zeigers zu den gefährlichsten Fehlern in der C gehört\autocite{TheCProgrammingLaunguage}.
\begin{figure}[htp]
    \centering
    \lstinputlisting[
        language=C,
        caption=Beispiel dangling pointer in C
    ]{img/dangling_pointer.c}
    \label{fig:dangling_pointer_in_c}
\end{figure}

\subsubsection{Double Free}
\label{sec:Double Free}
Wenn derselbe Zeiger mehrfach mit free() aufgerufen wird, kann dies die Heap
beschädigen und zu Laufzeitfehlern oder Sicherheitslücken führen. Solche Defekte 
sind insbesondere in Multithread-Umgebungen besonders gefährlich, da Angreifer 
durch sorgfältig gestaltete Eingaben double free ausnutzen können, 
um Heap Spraying durchzuführen oder beliebigen Code auszuführen.
\begin{figure}[htp]
    \centering
    \lstinputlisting[
        language=C,
        caption=Beispiel double free in C
    ]{img/tow_of_free.c}
    \label{fig:double_free_in_c}
\end{figure}

\subsubsection{Wild Pointers}
\label{sec:Wild Pointers}
Wenn ein Zeiger nach seiner Deklaration ohne Initialisierung verwendet wird, verweist 
er auf eine zufällige Adresse, was zu schwer vorhersehbarem Verhalten führen kann. 
Solche Fehler stellen ein ernstzunehmendes Sicherheitsrisiko dar.
\begin{figure}[htp]
    \centering
    \lstinputlisting[
        language=C,
        caption=Beispiel wild pointer in C
    ]{img/wild_pointer.c}
    \label{fig:wild_pointer_in_c}
\end{figure}

\subsubsection{Buffer Overflow}
\label{sec:Buffer Overflow}
Pufferüberläufe gehören zu den berüchtigtsten Sicherheitslücken in der Geschichte 
der Programmiersprache C. Da Standardbibliotheksfunktionen wie strcpy oder scanf 
die Grösse des Zielpuffers nicht überprüfen, kommt es leicht zu out of bounds Zugriffen. 
Angreifer können diese Schwachstelle ausnutzen, um den Stack zu manipulieren und den 
Programmsteuerfluss zu übernehmen\autocite{SmashingtheStackforFunandProfit}. \autocite{TheCProgrammingLaunguage}weisen 
bei der Beschreibung von String zwar prägnante Beispiele auf, verzichten jedoch auf 
Sicherheitsprüfungen der Grenzen – ein Umstand, der die unzureichende Berücksichtigung 
von Sicherheitsaspekten im frühen Design der C Library deutlich macht.
\begin{figure}[htp]
    \centering
    \lstinputlisting[
        language=C,
        caption=Beispiel wild pointer in C
    ]{img/wild_pointer.c}
    \label{fig:wild_pointer_in_c}
\end{figure}

\subsection{Rust}
\label{sec:Rust}
Gibt es eine Sprache, die so schnell ist wie C/C++, aber sicherer ist und 
viele typische Speicherfehler vermeidet?

Vor diesem Hintergrund wurde Rust im Jahr 2015 veröffentlicht mit dem Ziel, 
C/C++ als systemnahe Programmiersprache zu ersetzen und die Speichersicherheitsprobleme 
von C/C++ zu lösen, ohne dabei auf Leistung zu verzichten.

Rust verfolgt das Konzept der „Zero Cost Abstractions“\autocite{TheRustProgrammingLanguage}: Es behält eine Leistung bei, 
die mit C++ vergleichbar ist, bietet aber gleichzeitig fortschrittliche und sichere 
Programmierabstraktionen. 
Die Speicherverwaltung in Rust basiert auf drei Kernmechanismen: Ownership, Borrowing 
und Lifetimes. Jeder Wert kann zu einem bestimmten Zeitpunkt nur einen Besitzer haben, 
und sobald dieser Besitzer den Gültigkeitsbereich verlässt, wird der Wert automatisch 
freigegeben (Nach meinem Verständnis ist das ähnlich dem RAII-Prinzip in C++.). Wenn Datenzugriff erforderlich ist, 
geschieht dies über Leihen, wobei die Nutzung den Regeln der Lebensdauer unterliegt, damit die Daten 
während der Verwendung stets gültig bleiben. Der Compiler überprüft die Lebensdauern bereits zur 
Kompilierzeit und verhindert so hängende Zeiger oder ungültige Referenzen\autocite{TheRustProgrammingLanguage}. Darüber hinaus stellt Rust 
intelligente Zeiger wie Box, Vec, Rc, Arc bereit, die gängige Speicherverwaltungsmodelle 
kapseln und vereinfachen.

Im Gegensatz zu Go verwendet Rust Speicher automatisch durch sein Ownership-System,ohne Garbage Collection\autocite{TheRustProgrammingLanguage}, was die Leistung auf dem Niveau von C/C++ 
hält. Zudem bietet Rust ein starkes Typsystem, das viele Fehler bereits zur Kompilierzeit erkennt, 
und unterstützt funktionale Programmierparadigmen. Im Vergleich dazu kann C++ zwar ebenfalls durch 
Klassenkapselung „gefährlichen Code“ isolieren, erzwingt aber keine strikten Sicherheitsgrenzen Entwickler 
können jederzeit „über die Grenzen hinausgehen“. In Rust hingegen erzwingt der Compiler 
die Markierung von unsicherem Code und verlangt, dass externe APIs innerhalb des Typsystems sicher bleiben, 
was die Speicher und -Nebenlaufsicherheit erheblich verbessert.

So löst Rust das in C häufig auftretende Problem der „dangling Pointer:

\begin{figure}[htp]
    \centering
    \lstinputlisting[
        language=Rust,
        caption=Beispiel dangling pointer in Rust
    ]{img/dangling_pointer.rs}
    \label{fig:dangling_pointer_in_Rust}
\end{figure}

In Folgendem Code weist \textbf{Box::new(1)}einen i32 auf dem Heap zu und gibt einen Smart 
Pointer vom Typ Box<i32> zurück. Beim Ausführen von return num wird das Eigentum 
von num an den Aufrufer p übertragen, wodurch die Entstehung eines Dangling Pointers 
vermieden wird. Sobald p den Gültigkeitsbereich verlässt, wird automatisch das 
Drop-Trait von Box aufgerufen, um den belegten Heap-Speicher freizugeben und so 
eine sichere automatische Ressourcenverwaltung zu gewährleisten.

\begin{figure}[htp]
    \centering
    \lstinputlisting[
        language=Rust,
        caption= Beispiel: Behebung dangling pointer-Fehler aus C in Rust
    ]{img/dangling_pointer.rs}
    \label{fig:dangling_pointer_in_Rust}
\end{figure}

\begin{figure}[htp]
    \centering
    \lstinputlisting[
        language=Rust,
        caption=Beispiel: Behebung double free-Fehler aus C in Rust
    ]{img/tow_of_free.rs}
    \label{fig:double_free_in_Rust}
\end{figure}

In dem obigen Code allokiert vec![0; 10] einen zusammenhängenden Speicherbereich 
im Heap, um 10 <i32> Elemente zu speichern, während auf dem Stack lediglich ein 
Zeiger auf diesen Heap-Speicher, die Länge und die Kapazität gespeichert werden. 
Wenn data den Gültigkeitsbereich verlässt, ruft der Vec automatisch seinen 
Destruktor auf, um den belegten Heap-Speicher freizugeben und dadurch eine sichere 
Speicherverwaltung zu gewährleisten.

\subsubsection{Golang}
\label{sec:Golang}
Aufgrund der Komplexität und langsamen Buildungszeiten der C++ Projekte bei 
Google wurde eine einfachere, effizientere und nebenläufigkeitsfreundliche Sprache 
benötigt. Golang wurde 2007 von Google veröffentlicht. Die Designziele sind: einfache 
und leicht erlernbare Syntax durch Weglassen komplexer C++ Eigenschaften wie Vererbung, 
Templates und Makros; schnelle Entwicklung durch schnelle Kompilierung, die eine Iteration 
ähnlich wie bei Skriptsprachen ermöglicht; Nebenläufigkeit im Fokus mit eingebauten Goroutines 
und Channels, die parallele Programmierung erleichtern; und speziell geeignet für den Einsatz 
auf Serversseite zum Aufbau großskaliger verteilter Systeme.


\begin{figure}[htp]
    \centering
    \lstinputlisting[
        language=C,
        caption= Beispiel: Behebung dangling pointer-Fehler aus C in Golang
    ]{img/danling.go}
    \label{fig:dangling_pointer_in_Golang}
\end{figure}

Der obige Go-Code löst das in C häufig auftretende Problem mit Nullzeigern. 
Die Funktion dangling pointer gibt die Adresse der lokalen Variablen num zurück. 
Der Go-Compiler erkennt mithilfe der Escape-Analyse automatisch, dass die 
Adresse von num zurückgegeben wird, und verschiebt daher die Allokation 
von num vom Stack auf den Heap. Dadurch wird sichergestellt, dass der 
zurückgegebene Zeiger auf gültigen Speicher verweist. Im main Funktionsblock 
wird dieser Zeiger dereferenziert und ausgegeben, was korrekt den Wert 1 liefert 
und zeigt, dass der Zeiger sicher und gültig ist.
\begin{figure}[htp]
    \centering
    \lstinputlisting[
        language=C,
        caption= Beispiel: Behebung double free-Fehler aus C in Golang
    ]{img/tow_of_free.go}
    \label{fig:double_free_in_Golang}
\end{figure}

Go verwendet ein automatisches Garbage Collection System, das effektiv 
das in C durch manuelles Speichermanagement verursachte Problem des 
double free vermeidet, welches zu undefiniertem Verhalten oder sogar 
Programmabstürzen führen kann. Dadurch wird die Komplexität und das 
Risiko der Speicherverwaltung deutlich reduziert\autocite{TheGolangMemoryModel}. Im obigen Code wird 
der Speicher für das Slice mit make vom Go-Laufzeitsystem automatisch 
verwaltet. Indem die Variable auf nil gesetzt wird, wird die Referenz 
entfernt, sodass der Garbage Collector den nicht mehr genutzten Speicher 
zur geeigneten Zeit freigeben kann. Ein manuelles Aufrufen einer 
Freigabefunktion ist somit nicht erforderlich, wodurch das Risiko 
eines doppelten Freigebens vollständig ausgeschlossen wird.

Dieser Mechanismus nutzt die Escape Analyse des Go-Compilers und den nebenläufigen 
Garbage Collector zur Laufzeit. So wird sichergestellt, dass Variablen im richtigen 
Speicherbereich abgelegt und nach ihrer Lebenszeit sicher wieder freigegeben werden\autocite{TheGolangMemoryModel}. 
Dadurch macht die automatische Speicherverwaltung in Go die Entwicklung einfacher und sorgt ausserdem dafür, 
dass Programme stabiler und sicherer laufen – typische Speicherfehler, wie man sie aus anderen Sprachen kennt, 
werden so vermieden.

